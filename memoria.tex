\documentclass[12pt, a4paper]{article}

    \usepackage[spanish, es-tabla]{babel}
    \selectlanguage{spanish}
    \usepackage[utf8]{inputenc}
    \usepackage{listings}
    \usepackage{graphicx}
    \graphicspath{ {images/} }
    
    \lstset{
        basicstyle={\tiny\ttfamily},
        numbers=none,
        breaklines=true,
        breakatwhitespace=true
    }

    \begin{document}

        \title{% 
            \textbf{Pipeline para el Análisis en Tiempo Real de datos IoT} \\
            \large Trabajo de Investigación \\
            \textit{Internet de las Cosas en el Contexto de Big Data}\\
            \textit{Máster en Tecnologías de Análisis \\ de Datos Masivos: BIG DATA}
            }
        \author{Adrián Miralles Palazón}
        \date{\today}
        \maketitle

        \begin{figure}[h]
            \centerline{\includegraphics[]{logo}}
        \end{figure}

        \clearpage
        \tableofcontents
        
        \clearpage
        \section{Introducción}

        \paragraph{}
        El ecosistema de tecnologías desarrolladas para el análisis inteligente de datos producidos por dispositivos o sensores conectados, Internet de las Cosas (IoT), se encuentra en una fase de maduración e implantación en numerosas industrias. Casi todos los dispositivos que utilizamos se están convirtiendo en dispositivos inteligentes que están continuamente generando datos sobre parámetros internos o del medio que los rodea.

        \paragraph{}
        La recolección, el preprocesamiento, el almacenamiento y el análisis posterior de estos datos debe ser establecido mediante un procedimiento que permita la obtención de resultados clave para el dominio de aplicación del problema a solventar. Este procedimiento se conoce como \textit{data pipeline}.

        \paragraph{}
        


        \section{Motivación y objetivos}



        \section{Estado de arte}

        \subsection{Message Queuing Telemetry Transport (MQTT)}
        
        \subsection{Apache Kafka}
        
        \subsection{InfluxDB}
        
        \subsection{Grafana}
        


        \section{Diseño de solución}

        \subsection{Entidades}

        \subsubsection{Sensor + MQTT client (publisher)}
        \subsubsection{MQTT broker}
        \subsubsection{MQTT client (subscriber) + Kafka producer}
        \subsubsection{Kafka cluster}
        \subsubsection{Kafka consumer}
        \subsubsection{InfluxDB}
        \subsubsection{Grafana}

        \subsection{Interacción}
        


        \section{Desarrollo de solución}

        \subsection{Dispositivos y sensores}

        \subsection{Virtualización de servicios}
        
        \subsection{Desarrollo de servicios en Python}
        
        \subsection{Análisis de datos}
        
        \section{Conclusiones}

        % \begin{figure}[p]
        %         \centerline{\includegraphics[width=0.9\paperwidth]{ventas}}
        %         \caption{Modelo multidimensional de la tabla de hechos Ventas.}
        % \end{figure}

\end{document}